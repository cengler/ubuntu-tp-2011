
\section{Ejercicio 1}

Ejecutamos \verb|whoop| dentro del programa \verb|strace| para obterner informacion de las SYSCALLS del programa \verb|whoop|.

\begin{framed}
\begin{verbatim}
strace -f ./whoop ./tinyhello
\end{verbatim}
\end{framed}

Esto nos dio la siguiente salida.

\begin{framed}
\begin{verbatimtab}
execve("./whoop", ["./whoop", "./tinyhello"], [/* 39 vars */]) = 0
uname({sys="Linux", node="sullivan", ...}) = 0
brk(0)                                  = 0x9742000
brk(0x9742cd0)                          = 0x9742cd0
set_thread_area({entry_number:-1 -> 6, base_addr:0x9742830, limit:1048575, 
  seg_32bit:1, contents:0, read_exec_only:0, limit_in_pages:1, 
  seg_not_present:0, useable:1}) = 0
brk(0x9763cd0)                          = 0x9763cd0
brk(0x9764000)                          = 0x9764000
clone(Process 4252 attached child_stack=0, flags=CLONE_CHILD_CLEARTID|
  CLONE_CHILD_SETTID|SIGCHLD, child_tidptr=0x9742898) = 4252
[pid  4251] wait4(-1, Process 4251 suspended
 <unfinished ...>
[pid  4252] open("whoop.out", O_WRONLY|O_CREAT, 0644) = 3
[pid  4252] dup2(3, 1)                  = 1
[pid  4252] dup2(3, 2)                  = 2
[pid  4252] close(3)                    = 0
[pid  4252] rt_sigaction(SIGHUP, {SIG_IGN, [HUP], SA_RESTART}, 
  {SIG_DFL, [], 0}, 8) = 0
[pid  4252] execve("./tinyhello", ["./tinyhello"], [/* 39 vars */]) = 0
[pid  4252] write(1, "Hola SO!\n", 9)   = 9
[pid  4252] _exit(0)                    = ?
Process 4251 resumed
Process 4252 detached
<... wait4 resumed> [{WIFEXITED(s) && WEXITSTATUS(s) == 0}], 0, NULL) = 4252
--- SIGCHLD (Child exited) @ 0 (0) ---
exit_group(0)                           = ?
\end{verbatimtab}
\end{framed}

\subsection{Que hace whoop?}

\begin{enumerate}
 \item Abre el archivo \verb|whoop.out|
 \item Reemplaza el FD de STD-OUT el archivo \verb|whoop.out|
 \item Reemplaza el FD de STD-ERR el archivo \verb|whoop.out|
 \item Se reemplaza por proceso de \verb|whoop| con el contenido del programa pasado como parametro
\end{enumerate}


\subsection{Como lo hace whoop?}

Abre el archivo \verb|whoop.out|\\
\begin{framed}
\begin{verbatimtab}
open("whoop.out", O_WRONLY|O_CREAT, 0644) = 3
\end{verbatimtab}
\end{framed}

Este comando crea el archivo whoop.out en el direcctorio de ejecucion y retorna el FD 3

Luego se remplaza el FD de la salida de standart por el del archivo whoop.out

\begin{framed}
\begin{verbatimtab}
dup2(3, 1)                  = 1
\end{verbatimtab}
\end{framed}

y tambien se remplaza el FD de la salida de error por el del archivo whoop.out\\
\begin{framed}
\begin{verbatimtab}
dup2(3, 2)                  = 2
\end{verbatimtab}
\end{framed}

Se cierra el FD del archivo whoop (ya quedo direccionado por los FD de la STD-OUT y STD-ERR)\\
\begin{framed}
\begin{verbatimtab}
close(3)                    = 0
\end{verbatimtab}
\end{framed}

El proceso cambia la acci\'on de asociada a una se\~nal.
\begin{framed}
\begin{verbatimtab}
rt_sigaction(SIGHUP, {SIG_IGN, [HUP], SA_RESTART}, {SIG_DFL, [], 0}, 8) = 0
\end{verbatimtab}
\end{framed}

Reemplaza el la imagen del proceso corriente por la imagen del nuevo proceso pasado por parametro utilizando execve
\begin{framed}
\begin{verbatimtab}
execve("./tinyhello", ["./tinyhello"], [/* 39 vars */]) = 0
\end{verbatimtab}
\end{framed}

Codigo del nuevo porceso proceso (tinyhello)
\begin{framed}
\begin{verbatimtab}
write(1, "Hola SO!\n", 9)   = 9
_exit(0)                    = ?
\end{verbatimtab}
\end{framed}
