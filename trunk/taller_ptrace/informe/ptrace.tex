
\section{Ejercicio 1}

Ejecutamos \verb|whoop| dentro del programa \verb|strace| para obterner informacion de las SYSCALLS del programa \verb|whoop|.

\begin{framed}
\begin{verbatim}
strace -f ./whoop ./tinyhello
\end{verbatim}
\end{framed}

Esto nos di\'o la siguiente salida.

\begin{framed}
\begin{verbatimtab}
execve("./whoop", ["./whoop", "./tinyhello"], [/* 39 vars */]) = 0
uname({sys="Linux", node="sullivan", ...}) = 0
brk(0)                                  = 0x9742000
brk(0x9742cd0)                          = 0x9742cd0
set_thread_area({entry_number:-1 -> 6, base_addr:0x9742830, limit:1048575, 
  seg_32bit:1, contents:0, read_exec_only:0, limit_in_pages:1, 
  seg_not_present:0, useable:1}) = 0
brk(0x9763cd0)                          = 0x9763cd0
brk(0x9764000)                          = 0x9764000
clone(Process 4252 attached child_stack=0, flags=CLONE_CHILD_CLEARTID|
  CLONE_CHILD_SETTID|SIGCHLD, child_tidptr=0x9742898) = 4252
[pid  4251] wait4(-1, Process 4251 suspended
 <unfinished ...>
[pid  4252] open("whoop.out", O_WRONLY|O_CREAT, 0644) = 3
[pid  4252] dup2(3, 1)                  = 1
[pid  4252] dup2(3, 2)                  = 2
[pid  4252] close(3)                    = 0
[pid  4252] rt_sigaction(SIGHUP, {SIG_IGN, [HUP], SA_RESTART}, 
  {SIG_DFL, [], 0}, 8) = 0
[pid  4252] execve("./tinyhello", ["./tinyhello"], [/* 39 vars */]) = 0
[pid  4252] write(1, "Hola SO!\n", 9)   = 9
[pid  4252] _exit(0)                    = ?
Process 4251 resumed
Process 4252 detached
<... wait4 resumed> [{WIFEXITED(s) && WEXITSTATUS(s) == 0}], 0, NULL) = 4252
--- SIGCHLD (Child exited) @ 0 (0) ---
exit_group(0)                           = ?
\end{verbatimtab}
\end{framed}

\subsection{Que hace whoop?}

\begin{enumerate}
 \item Crea un proceso hijo dentro del cual se va a ejecutar el programa pasado como par\'ametro.
 \item El padre queda en espera a que el hijo termine de ejecutar.
 \item el hijo:
 \begin{enumerate}
  \item Abre el archivo \verb|whoop.out|
  \item Reemplaza el FD de STD-OUT el del archivo \verb|whoop.out|
  \item Reemplaza el FD de STD-ERR el del archivo \verb|whoop.out|
  \item Desasocia el proceso hijo, para que en caso de que el padre muera, el hijo continue en ejecucion.
  \item Se reemplaza por proceso de \verb|whoop| con el contenido del programa pasado como parametro
 \end{enumerate}
\end{enumerate}

\newpage

\subsection{Como lo hace whoop?}

Crea un proceso hijo dentro del cual se va a ejecutar el programa pasado como par\'ametro.

\begin{framed}
\begin{verbatimtab}
clone(Process 4252 attached child_stack=0, flags=CLONE_CHILD_CLEARTID|
  CLONE_CHILD_SETTID|SIGCHLD, child_tidptr=0x9742898) = 4252
\end{verbatimtab}
\end{framed}

El padre queda en espera a que el hijo termine de ejecutar. 

\begin{framed}
\begin{verbatimtab}
[pid  4251] wait4(-1, Process 4251 suspended
\end{verbatimtab}
\end{framed}

En hijo: Abre el archivo \verb|whoop.out|\\
Este comando crea el archivo whoop.out en el directorio de ejecuci\'on y retorna el FD 3.\\
Luego se remplaza el FD de la salida de standart y la salida de error por el del archivo whoop.out\\
y cierra el FD del archivo whoop (ya quedo direccionado por los FD de la STD-OUT y STD-ERR)

\begin{framed}
\begin{verbatimtab}
open("whoop.out", O_WRONLY|O_CREAT, 0644) = 3
dup2(3, 1)                  = 1
dup2(3, 2)                  = 2
close(3)                    = 0
\end{verbatimtab}
\end{framed}

El proceso cambia la acci\'on de asociada a una se\~nal. HUP (exit)

La llamada al sistema\\
 \verb|long sys_rt_sigaction (int sig, const struct sigaction |\\
 \verb|                   *act, struct sigaction *oact, size_t sigsetsize);| \\
permite realizar eso.

Altera la accion de la se\~nal: \verb|sig| y le asigna el comportamiento definido en la estructura apuntada por \verb|act| y salva la accion original en la estructura \verb|oact|.
\verb|sigsetsize| indica el tama\~no de la estuctura apuntada por \verb|act|.

En este caso se esta cambiando la accion de la se\~nal \verb|SIGHUP|, y se le esta asignando el comportamiento \verb|SIG_IGN| que es ignorar la se\~nal de \verb|SIGHUP|.

\begin{framed}
\begin{verbatimtab}
rt_sigaction(SIGHUP, {SIG_IGN, [HUP], SA_RESTART}, {SIG_DFL, [], 0}, 8) = 0
\end{verbatimtab}
\end{framed}

Reemplaza la imagen del proceso corriente por la imagen del nuevo proceso pasado por par\'ametro utilizando \verb|execve|
\begin{framed}
\begin{verbatimtab}
execve("./tinyhello", ["./tinyhello"], [/* 39 vars */]) = 0
\end{verbatimtab}
\end{framed}

Codigo del nuevo proceso (tinyhello)
\begin{framed}
\begin{verbatimtab}
write(1, "Hola SO!\n", 9)   = 9
_exit(0)                    = ?
\end{verbatimtab}
\end{framed}

El proceso hijo finaliza y luego el padre finaliza el terminar su \verb|wait4()|. 	

\newpage

\section{Ejercicio 2}

Para poder modificar los datos que fuera a escribir por pantalla el proceso hijo, debiamos interceptar las llamdas al SO que hacia el hijo.\\
Para eso utilizamos el tip con el codigo para interceptar las llamadas del SO que hace el proceso hijo.


Discriminamos que llamadas del sistema operativo eran de WRITE y de estas cuales eran a STDOUT.\\
Para eso utilizamos las llamada al SO que permite espiar los registros del proceso hijo suspendido

\begin{framed}
\begin{verbatimtab}
int sysno = ptrace(PTRACE_PEEKUSER, child, 4*ORIG_EAX, NULL);
int destino = ptrace(PTRACE_PEEKUSER, child, 4*EBX, NULL);
\end{verbatimtab}
\end{framed}

Si la llamada era efectivamente una llamade a WRITE entonces ECX era el puntero a los datos y EDX la longitud de los datos a imprimir.

\begin{framed}
\begin{verbatimtab}
int addr = ptrace(PTRACE_PEEKUSER, child, 4*ECX, NULL);
int len = ptrace(PTRACE_PEEKUSER, child, 4*EDX, NULL);	
\end{verbatimtab}
\end{framed}

Ahora debiamos cambiar los datos que se enviarian a imprimir por pantalla.\\
Para eso utilizamos \verb|PTRACE_PEEKDATA| para obtener los datos de direcciones de memoria del proceso hijo.

\begin{framed}
\begin{verbatimtab}
long data = ptrace(PTRACE_PEEKDATA, child, addr+WORD_SIZE*i, NULL);
\end{verbatimtab}
\end{framed}

Reemplazamos el texto obtenido a mayusculas y lo teniamos que escribir en la memoria del hijo para que al continuar imprima el texto modificado.

\begin{framed}
\begin{verbatimtab}
ptrace(PTRACE_POKEDATA, child, addr+WORD_SIZE*i, data.val);
\end{verbatimtab}
\end{framed}

Esto lo hicimos hasta la longitud del texto. Y luego llamamos a ptrace(PTRACE CONT) para reanudar el hijo hasta la siguiente se \~nal.
