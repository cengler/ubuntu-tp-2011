\documentclass[a4paper, 10pt, notitlepage]{article}

\usepackage{moreverb} %para importar codigo

\usepackage[spanish,activeacute]{babel}
\usepackage{babel} %paquete de idioma

\usepackage[latin1]{inputenc}

\usepackage{pepotina} %caratula
%\usepackage{color}

%\usepackage{hyperref}
%\usepackage[all]{hypcap}

\usepackage{fancyhdr} %linea sup con comentarios

\usepackage{lscape} %para hoja apaisada

\usepackage{framed} %para crear cajas de texto

\usepackage{lastpage} %ultima pagina

%\usepackage{wrapfig} %Inclusi�n de gr�ficos al lado de texto
%\usepackage[rflt]{floatflt} %Para meter figuras flotantes entre el texto


%\usepackage{pstricks}
%\usepackage{uml} %UML

\usepackage{listings}
%\lstset{
%  breaklines=true,                                     % line wrapping on
%  language=ocl,
%  frame=ltrb,
%  framesep=5pt,
%  basicstyle=\normalsize,
%  keywordstyle=\ttfamily\color{OliveGreen},
%  identifierstyle=\ttfamily\color{CadetBlue}\bfseries,
%  commentstyle=\color{Brown},
%  stringstyle=\ttfamily,
%  showstringspaces=ture
%}

\addtolength{\topmargin}{-50pt} 
\addtolength{\textwidth}{105pt}
\addtolength{\textheight}{120pt}
\addtolength{\oddsidemargin}{-50pt}

%\newcommand{\minix}{\textsl{minix }}

%%% Encabezado y pie de p'agina
\pagestyle{fancy}
\fancyhead[LO]{Sistemas Operativos}
\fancyhead[C]{}
\fancyhead[RO]{P\'agina \thepage\ de \pageref{LastPage}}
\renewcommand{\headrulewidth}{0.4pt}
\fancyfoot{}

\newcommand{\depto}{{\bf DPTO: }}


\def\falta#1{ \begin{framed}	\begin{center} \hspace{1cm} \Large FALTA \normalsize #1 \hspace{1cm} \end{center} \end{framed}}


\def\imagen#1#2{\vskip0.5cm
\begin{center}
\includegraphics[scale=#1]{#2}
\end{center}}


\begin{document}

\universidad{Universidad de Buenos Aires}
\facultad{Facultad de Ciencias Exactas y Naturales}
\departamento{Departamento de Computaci�n}
\materia{Sistemas Operativos}
\resumen{Realizar ingenier reversa cobre el programa compilado whoop y explicar en lenguaje de alto nivel lo que hace y como los hace.
tambien desarrollar el programa upcase en C que ejecute el comando pasado por parametro, pero haciendo que toda la salida estandar del programa aparezca en mayusculas.}
\keys{syscalls, ptrace}
\titulo{Taller: ptrace}
\subtitulo{taller de syscalls}
\grupo{Numero de grupo: 12}
\fecha{1er Cuatrimeste 2011}
\footspace{1cm}
\integrante{Dominguez, Pablo Sebastian}{108/05}{pablo.sebastian.dominguez@gmail.com}
\integrante{Engler, Christian Alejandro}{314/05}{caeycae@gmail.com}

%caratula	
\maketitle{}

\tableofcontents

\newpage

\section{Ejercicio 1}

Ejecutamos \verb|whoop| dentro del programa \verb|strace| para obterner informacion de las SYSCALLS del programa \verb|whoop|.

\begin{framed}
\begin{verbatim}
strace -f ./whoop ./tinyhello
\end{verbatim}
\end{framed}

Esto nos dio la siguiente salida.

\begin{framed}
\begin{verbatimtab}
execve("./whoop", ["./whoop", "./tinyhello"], [/* 39 vars */]) = 0
uname({sys="Linux", node="sullivan", ...}) = 0
brk(0)                                  = 0x9742000
brk(0x9742cd0)                          = 0x9742cd0
set_thread_area({entry_number:-1 -> 6, base_addr:0x9742830, limit:1048575, 
  seg_32bit:1, contents:0, read_exec_only:0, limit_in_pages:1, 
  seg_not_present:0, useable:1}) = 0
brk(0x9763cd0)                          = 0x9763cd0
brk(0x9764000)                          = 0x9764000
clone(Process 4252 attached child_stack=0, flags=CLONE_CHILD_CLEARTID|
  CLONE_CHILD_SETTID|SIGCHLD, child_tidptr=0x9742898) = 4252
[pid  4251] wait4(-1, Process 4251 suspended
 <unfinished ...>
[pid  4252] open("whoop.out", O_WRONLY|O_CREAT, 0644) = 3
[pid  4252] dup2(3, 1)                  = 1
[pid  4252] dup2(3, 2)                  = 2
[pid  4252] close(3)                    = 0
[pid  4252] rt_sigaction(SIGHUP, {SIG_IGN, [HUP], SA_RESTART}, 
  {SIG_DFL, [], 0}, 8) = 0
[pid  4252] execve("./tinyhello", ["./tinyhello"], [/* 39 vars */]) = 0
[pid  4252] write(1, "Hola SO!\n", 9)   = 9
[pid  4252] _exit(0)                    = ?
Process 4251 resumed
Process 4252 detached
<... wait4 resumed> [{WIFEXITED(s) && WEXITSTATUS(s) == 0}], 0, NULL) = 4252
--- SIGCHLD (Child exited) @ 0 (0) ---
exit_group(0)                           = ?
\end{verbatimtab}
\end{framed}

\subsection{Que hace whoop?}

\begin{enumerate}
 \item Abre el archivo \verb|whoop.out|
 \item Reemplaza el FD de STD-OUT el del archivo \verb|whoop.out|
 \item Reemplaza el FD de STD-ERR el del archivo \verb|whoop.out|
 \item Se reemplaza por proceso de \verb|whoop| con el contenido del programa pasado como parametro
\end{enumerate}

\newpage

\subsection{Como lo hace whoop?}

Abre el archivo \verb|whoop.out|\\
Este comando crea el archivo whoop.out en el direcctorio de ejecucion y retorna el FD 3

\begin{framed}
\begin{verbatimtab}
open("whoop.out", O_WRONLY|O_CREAT, 0644) = 3
\end{verbatimtab}
\end{framed}

Luego se remplaza el FD de la salida de standart por el del archivo whoop.out

\begin{framed}
\begin{verbatimtab}
dup2(3, 1)                  = 1
\end{verbatimtab}
\end{framed}

y tambien se remplaza el FD de la salida de error por el del archivo whoop.out
\begin{framed}
\begin{verbatimtab}
dup2(3, 2)                  = 2
\end{verbatimtab}
\end{framed}

Se cierra el FD del archivo whoop (ya quedo direccionado por los FD de la STD-OUT y STD-ERR)
\begin{framed}
\begin{verbatimtab}
close(3)                    = 0
\end{verbatimtab}
\end{framed}

El proceso cambia la acci\'on de asociada a una se\~nal.
\begin{framed}
\begin{verbatimtab}
rt_sigaction(SIGHUP, {SIG_IGN, [HUP], SA_RESTART}, {SIG_DFL, [], 0}, 8) = 0
\end{verbatimtab}
\end{framed}

Reemplaza el la imagen del proceso corriente por la imagen del nuevo proceso pasado por parametro utilizando execve
\begin{framed}
\begin{verbatimtab}
execve("./tinyhello", ["./tinyhello"], [/* 39 vars */]) = 0
\end{verbatimtab}
\end{framed}

Codigo del nuevo porceso proceso (tinyhello)
\begin{framed}
\begin{verbatimtab}
write(1, "Hola SO!\n", 9)   = 9
_exit(0)                    = ?
\end{verbatimtab}
\end{framed}

\newpage

\section{Ejercicio 2}

Escribimos en programa upcase.c

\begin{framed}
\begin{verbatimtab}
int main(int argc, char* argv[]) {
	
	if(argc < 2 )
	{

	}

	/* Hacemos fork, para tener un proceso controlador y el que nos pasan por parametro */
	int child = fork();
	
	/* Si hubo error en el fork */
	if (child == -1) { 
		perror("ERROR fork"); 
		return 1; 
	}
	
	if (child == 0) { /* Si ej el hijo */

		/* Lo preparo para ser monitoreado */
		if (ptrace(PTRACE_TRACEME, 0, NULL, NULL)) {
			perror("ERROR child ptrace(PTRACE_TRACEME, ...)"); exit(1);
		}
		
		/* Ejecuto el programa pasado por parametro */
		execvp(argv[1], argv+1);
		
		/* Si sigue despues del exec es porque hubi un error */
		perror("ERROR child exec(...)"); 
		exit(1);	
	
	} else { /* Si es el padre */
		
		/* Mientras que el proceso no salga, ciclamos infinitamente */
		while(1) {
			
			/* El proceso padre se bloquea hasta que el hijo tenga un evento */
			int status;
			if (wait(&status) < 0) { 
				perror("wait"); 
				break; 
			}
			
			/* Si el hijo finalizo corto el ciclo infinito */
			if (WIFEXITED(status)) 
				break; /* Proceso terminado */
			
			/* Verificamos y modificamos la memoria y el hijo bloqueado */
			editBloquedProcess(child);
			
			/* Liberamos al proceso hijo para que continue ejecutando */
			ptrace(PTRACE_SYSCALL, child, NULL, NULL);
		}
		
		/* Liberamos al hijo para que continue se ejecucion */
		ptrace(PTRACE_DETACH, child, NULL, NULL);
	}
	return 0;
}
\end{verbatimtab}
\end{framed}

\end{document}


