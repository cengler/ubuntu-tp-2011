\documentclass[a4paper, 10pt, notitlepage]{article}

\usepackage{moreverb} %para importar codigo

\usepackage[spanish,activeacute]{babel}
\usepackage{babel} %paquete de idioma

\usepackage[latin1]{inputenc}

\usepackage{pepotina} %caratula
%\usepackage{color}

\usepackage{hyperref}
%\usepackage[all]{hypcap}

\usepackage{fancyhdr} %linea sup con comentarios

\usepackage{lscape} %para hoja apaisada

\usepackage{framed} %para crear cajas de texto

\usepackage{lastpage} %ultima pagina

%\usepackage{pstricks}
%\usepackage{uml} %UML

\usepackage{listings}
%\lstset{
%  breaklines=true,                                     % line wrapping on
%  language=ocl,
%  frame=ltrb,
%  framesep=5pt,
%  basicstyle=\normalsize,
%  keywordstyle=\ttfamily\color{OliveGreen},
%  identifierstyle=\ttfamily\color{CadetBlue}\bfseries,
%  commentstyle=\color{Brown},
%  stringstyle=\ttfamily,
%  showstringspaces=ture
%}

\addtolength{\topmargin}{-50pt} 
\addtolength{\textwidth}{105pt}
\addtolength{\textheight}{120pt}
\addtolength{\oddsidemargin}{-50pt}

%\newcommand{\minix}{\textsl{minix }}

%%% Encabezado y pie de p'agina
\pagestyle{fancy}
\fancyhead[LO]{Ingenieria del Software I}
\fancyhead[C]{}
\fancyhead[RO]{P\'agina \thepage\ de \pageref{LastPage}}
\renewcommand{\headrulewidth}{0.4pt}
\fancyfoot{}

\newcommand{\depto}{{\bf DPTO: }}


\def\falta#1{ \begin{framed}	\begin{center} \hspace{1cm} \Large FALTA \normalsize #1 \hspace{1cm} \end{center} \end{framed}}


\def\imagen#1#2#3{\vskip0.5cm
{\large #3}
\begin{center}
\includegraphics[scale=#1]{#2}
\end{center}}


\begin{document}

\universidad{Universidad de Buenos Aires}
\facultad{Facultad de ciencias exactas y naturales}
\departamento{Departamento de Computacion}
\materia{Sistemas Operativos}
\resumen{Simusched}
\keys{Scheduling, c++}
\titulo{Tp1: Simusched}
\subtitulo{Scheduling de tareas}
\grupo{Numero de grupo: 12}
\fecha{1er Cuatrimeste 2011}
\footspace{1cm}
\integrante{Dominguez, Pablo Sebastian}{000/05}{pablo.sebastian.dominguez@gmail.com}
\integrante{Engler, Christian Alejandro}{314/05}{caeycae@gmail.com}

%caratula
\maketitle{}

\tableofcontents

%\newpage

\part{Parte 1: Entendiendo el Simulador simusched}
\section{Introducción}

\subsection{Ejercicio 1}

\paragraph{Implementamos el Metodo TaskCon segun lo pedido.}

\begin{framed}
\begin{verbatim}
void TaskCon(vector<int> params) { // params: n
	int n = params[0];
	int bmin = params[1];
	int bmax = params[2];
	for( int i = 0; i < n ; i++ ) {
		int time = (rand()%(bmax-bmin))+bmin;
		uso_IO(time);
	}
}
\end{verbatim}
\end{framed}

\subsection{Ejercicio 2}

\paragraph{Creamos el archivo \verb|ejercicio2.tsk|}

\begin{framed}
\begin{verbatim}
# FILE ejercicio2.tsk
TaskCPU 10
TaskCPU 10
TaskCom 7 3 3
\end{verbatim}
\end{framed}

\paragraph{
Luego ejecutamos Simusched con los siguientes parametros:

\verb+./simusched ejercicio2.tsk 1 SchedFCFS | python graphsched.py > ejercicio2.png+
}

\imagen{0.5}{img/ejercicio2.png}{Scheduling FCFS}


%\subsection{Modulos Server}
%\newpage
%\paragraph{Descripcion de cada uno de los modulos}
%\label{sec:DescripcionDeCadaUnoDeLosModulos}
%\subsubsection{Clausura del casino}
%\newpage
%\imagen{0.5}{img/abrirCasino.png}{Abrir Casino}
%\input{introduccion.tex}
%\begin{framed}
%\begin{verbatim}
%<?xml version="1.0" encoding="UTF-8" ?> 
%</cerrarCasino>	
%\end{verbatim}
%\end{framed}

\end{document}


